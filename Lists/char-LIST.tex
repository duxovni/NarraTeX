%%%%%
%%
%% This file sets up the Char, PC, NPC, and Name datatypes and creates
%% macros for each.  These are for characters in the game.  Here you
%% use the fields in Char to assign elements to each character.
%%
%%
%%
%% \MYname (and the player name) is parsed by \parsename, the command
%% provided by LaTeX/parsename.sty.  See that file and
%% Extras/README-namemappings for ways to take advantage of this.
%%
%%
%%
%% \MYsex is set to either \male, \female, or \nonbinary, as correct
%% for the character.  \mfn{<masculine>}{<feminine>}{<nonbinary>} will
%% produce the correct form based on the current value
%% of \MYsex.  \mfn should only be used within the scope of a Char
%% macro.
%%
%% \pronoun{<command>}{<masc>}{<fem>}{<enby>} makes <command> a
%% wrapper around \mfn.  It is used to create a list of
%% gender-sensitive macros, mostly pronouns.  For example, given
%% \pronoun{\They}{He}{She}{They}, \cJamesBond{\They} will produce He.
%%
%%
%%
%% \badgetrue and \badgefalse toggle whether or not a Char macro will
%% produce a namebadge.
%%
%% \statstrue and \statsfalse will toggle the statcard.
%%
%% \skillstrue and \skillsfalse will toggle the skill list.  The skill
%% list prints both skills and stats (even if \statsfalse is set).
%%
%% \sheettrue and \sheetfalse will toggle the character sheet.
%%
%% \listtrue and \listfalse toggle whether the Char macro can display
%% in the playerlist.
%%
%% \labeltrue and \labelfalse toggle the box label.
%%
%%
%%
%% Some of the Char datatype's setup is in LaTeX/gametex.sty, to keep
%% this file short.
%%
%%%%%


%%%%%
%% If a field is declared by \F, it must be given a value by \s inside
%% \NEW, even if it's blank.  If you want it to be optional, declare
%% it with \FD<field> {<default>} here.
%%
%% Use \newstat to create stats (below, in \PRESETS{Char}).  The
%% <default> value is used unless the given Char macro sets the field.
%% For example:
%%
%%   \newstat\MYhp	{Hit Points}{HP}{5}
%%
%% would give character a Hit Points stat, abbreviated HP, referenced
%% as the \MYhp field, that defaults to 5.
\PRESETS{Char}{
  \FD\MYdesc	{} %% badge description

  \newstat\MYcr	{Combat Rating}{CR}{2} %% for DarkWater-style combat

  \FD\MYsex	{\nonbinary} %% \male, \female, \nonbinary
  \FD\MYrole	{} %% playerlist role
  \FD\MYgroupstr{} %% playerlist groupstring
  \FD\MYfile	{} %% character sheet filename (including .tex)
  \FS\MYtext	{\ifx\MYfile\empty\else%
		  \getextractenvs{document}{\chars/\MYfile}%
		\fi}
  \FD\MYwrapup	{}
  \badgetrue\statstrue\skillstrue\sheettrue\listtrue\labeltrue
  }

\POSTSETS{Char}{
  \resolvestats
  }


%%%%%
%% pronouns and similar gender-based macros
%%
%% \male, \female, \nonbinary
%% \mfn{<masculine>}{<feminine>}{<nonbinary>}
%% \pronoun{<command>}{<masculine>}{<feminine>}{<nonbinary>}
\def\male{0}\def\female{1}\def\nonbinary{2}
\newcommand{\mfn}[3]{\ifcase\MYsex#1\or#2\else#3\fi}
\newcommand{\pronoun}[4]{\def#1{\mfn{#2}{#3}{#4}}}

\pronoun{\they}		{he}{she}{they}
\pronoun{\They}		{He}{She}{They}
\pronoun{\them}		{him}{her}{them}
\pronoun{\Them}		{Him}{Her}{Them}
\pronoun{\their}	{his}{her}{their}
\pronoun{\Their}	{His}{Her}{Their}
\pronoun{\theirs}	{his}{hers}{theirs}
\pronoun{\Theirs}	{His}{Hers}{Theirs}
\pronoun{\themself}	{himself}{herself}{themself}
\pronoun{\Themself}	{Himself}{Herself}{Themself}
\pronoun{\spouse}	{husband}{wife}{spouse}
\pronoun{\Spouse}	{Husband}{Wife}{Spouse}
\pronoun{\offspring}	{son}{daughter}{child}
\pronoun{\Offspring}	{Son}{Daughter}{Child}
\pronoun{\kid}		{boy}{girl}{kid}
\pronoun{\Kid}		{Boy}{Girl}{Kid}
\pronoun{\sibling}	{brother}{sister}{sibling}
\pronoun{\Sibling}	{Brother}{Sister}{Sibling}
\pronoun{\parent}	{father}{mother}{parent}
\pronoun{\Parent}	{Father}{Mother}{Parent}
\pronoun{\pibling}	{uncle}{aunt}{pibling}
\pronoun{\Pibling}	{Uncle}{Aunt}{Pibling}
\pronoun{\nibling}	{nephew}{niece}{nibling}
\pronoun{\Nibling}	{Nephew}{Niece}{Nibling}
\def\aunt{\pibling}
\def\Aunt{\Pibling}
\def\uncle{\pibling}
\def\Uncle{\Pibling}
\def\niece{\nibling}
\def\Niece{\Nibling}
\def\nephew{\nibling}
\def\Nephew{\Nibling}
\pronoun{\person}	{man}{woman}{person}
\pronoun{\Person}	{Man}{Woman}{Person}
\pronoun{\sex}		{male}{female}{non-binary}
\pronoun{\Sex}		{Male}{Female}{Non-binary}


%%%%%
%% PC is a subtype of Char, for regular PCs.
\DECLARESUBTYPE{PC}{Char}
\PRESETS{PC}{\sd\MYgroupstr{pc}}


%%%%%
%% NPC is a subtype of Char.
\DECLARESUBTYPE{NPC}{Char}
\PRESETS{NPC}{\sd\MYgroupstr{npc}}


%%%%%
%% Name is a subtype of NPC.
%% For an in-text name.  By default, produces no packet material.
\DECLARESUBTYPE{Name}{Char}
\PRESETS{Name}{
  \badgefalse\statsfalse\skillsfalse\sheetfalse\listfalse\labelfalse
  \sd\MYgroupstr{name}
  }


%%%%%
%% support code for point skills, which appear on statcards.  This is
%% a useful alternative to abilities (or skills) for things expressed
%% as a simple level or number of points.
%%
%% Once declared, a point skill can be: 
%%
%%   invoked in a character's definition to give that character the
%%   skill (e.g. given \perdayskill\engineering{Engineering},
%%   use \engineering{2} in a Char to give them "Engineering: 2/day"
%%   on the statcard)
%%
%%   or invoked in game text to show a number of points
%%   (e.g. \engineering{2} in game text for "Engineering 2"), for
%%   requirements and such.
%%
%% Use \newpointskilltype{<command>}{<string>} to declare a type of
%% point skill.  E.g. \newpointskilltype{\perdayskill}{/day} to create
%% the \perdayskill{<skill>} command, which in turn creates per-day
%% skills.
\def\newpointskilltype#1#2{
  \def#1##1##2{%
    \def##1{\@skill{##2}}%
    \PRESETS{Char}{\@setval##1{-1}\def##1{\@pointstat{##1}{#2}{##2}}}%
    \POSTSETS{Char}{\def##1{\@skill{##2}}}%
    }
  }
\def\@pointstat#1#2#3#4{%
  \append\MYstats{\stat{#3}{#3}{#4#2}}%
  \@setval#1{#4}}
\def\@skill#1#2{#1\ifx\relax#2\relax\else\space #2\fi}
\def\@setval#1#2{\expandafter\def\csname\string#1\endcsname{#2}}
\def\@skillvalue#1{\csname\string#1\endcsname}

%% per-day skills are for N points per day.
\newpointskilltype{\perdayskill}{/day}

\perdayskill\engineering{Engineering}
\perdayskill\security{Security}

%% If a character hasn't been given a point skill, they will still
%% have a value (of -1) accessible.  This is useful for tests
%% (in \POSTSETS) where you want do things like assign greensheets to
%% people with points of the skill (and let value 0 mean they get the
%% greensheet but have no points).

%\POSTSETS{PC}{
%  \ifnum\@skillvalue\security>-1\relax
%    \append\MYgreens	{\gSecurity{}}
%  \fi
%  }


%%%%%%%%%%%%%%%%%%%%%%%%%%%%%%%%%%%%%%%%%%%%%%%%%%%%%%%%%%%%%%%%%%

%% don't use \cTest as a copy-and-paste template to populate your
%% character list.  Use something simpler, like
%%
%%   \NEW{PC}{\cBlah}{
%%     \s\MYname	{}
%%     \s\MYfile	{}
%%     }
%%
\NEW{PC}{\cTest}{
  \s\MYname	{Test Character}
  \s\MYfile	{README.tex}
  \s\MYnumber	{00000}
  \s\MYdesc	{a test}
  \s\MYplayer	{Test Player}
  \s\MYemail	{test@test.test}
  \s\MYaddress	{Test, rm 000}
  \s\MYphone	{x0-0000}
  \s\MYblues	{\bTest{}}
  \s\MYgreens	{\gTest{}\nGreenTest{}}
  \s\MYabils	{\aTest{}
		\aTestCombat{}
		}
  \engineering	{2}
  \s\MYitems	{\iTest{}\nTest{}}
  \s\MYwhites	{\wTest{}}
  \s\MYcash	{\cash{Dollar}{261}}
  \s\MYwrapup	{MIT cruft, originally from London, who knows some people.}
  }

\NEW{NPC}{\cNPC}{
  \s\MYname	{Nathan Clueless \suf III}
  \nickname	{Nate Clueless}
  \s\MYnumber	{00000}
  \s\MYdesc	{a suspicious person}
  \s\MYplayer	{GM Helper}
  }

\NEW{Name}{\cSomeGuy}{
  \maptrueformal %% most call him Sir Not-Appearing
  \s\MYname	{Sir \pre Robert Not-Appearing}
  }


%%%%%%%%%%%%%%%%%%%%%%%%%%%%%%%%%%%%%%%%%%%%%%%%%%%%%%%%%%%%%%%%%%
